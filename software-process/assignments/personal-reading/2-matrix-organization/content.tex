\subsubsection*{}
The first guest lecture during the software process course from a Spotify employee made me curious for the mechanics of a matrix organization so I wanted to know more about it. Although this is more a business-related topic, I found it interesting how the matrix organization fits an organization such as Spotify.

\subsection*{What is a Matrix Organization?}
A matrix organization is intermediate organizational form between functional and product organizations \autocite[70]{MEE196470}.
TO establish both forms in a matrix organization there is a notion of `Dual Authority' \autocite[36]{Galbraith197129}.
A matrix organization originates from the 1960's when there was need for both technical competences and capacity to deliver in the aerospace industry \autocite[30, 39]{Galbraith197129}.
A matrix organization is organized in a both horizontal and vertical manner. In such that staff members are located in functional vertical slices and in horizontal project based slices. \autocite[70-71]{MEE196470}
The purpose of the vertical slice is to create functional expertise and to facilitate ``the acquisition of specialized input'' \autocite[30]{Galbraith197129}.
The horizontal slice, on the other hand, facilitates on-time completion of projects due to the allocation of resources \autocite[30]{Galbraith197129}.
Managers with different skill sets manage the different types of slices. Managers that cope with staff quality and development of staff manage the vertical slices. The horizontal slices are managed to facilitate decision making for the success of the project. \autocite[31]{Galbraith197129}

There is no absolute form of a matrix organization \autocite[117]{KNIGHT} \autocite[31-35]{Galbraith197129}. On a spectrum of organizational types, from functional organizations to project organizations, a matrix organization would lie somewhere in the middle. Variation in a matrix organization may be achieved to implement the organization to the left or to the right of this spectrum. Staff members can for example participate in both the horizontal and vertical slice during certain projects, but not full-time \autocite[114]{KNIGHT}.

\subsection*{Why do we need Matrix Organizations?}
When an organization consists of functional divisions, it can empower technical specialty. However, it is hard to organize that a division will deliver on schedule for another division that is depending on this work \autocite[30]{Galbraith197129}.
Project organizations solve this problem, but tend to have problems on resource acquisition and have specialized personnel \autocite[30]{Galbraith197129}.
In small organizations these problems have no significant effect, but as organizations grow, these problems become of greater importance to solve to maintain a market position.

The goal behind the matrix organization is to intertwine both organization forms to eliminate the problems such as those sketched above.

A great advantage of the matrix organization is that it both enables an organization to have specialized personnel, and to be more fluid and to adapt to market demands \autocite[31]{Galbraith197129}.
However, a disadvantage of the matrix organization is the additional managed overhead \autocite[34]{Galbraith197129}. Both horizontal and vertical slices need to be managed. This results in additional personnel and thus additional costs.

Due to the fact that employees have to deal with dual authority, a functional and a project manager, the number of communication lines will increase \autocite[38]{Galbraith197129}. This may result in more meetings and can make the sharing of knowledge more complex.

Another aspect introduced by a matrix organization is the increasing complexity in the organization.  There are more people with the authority to make decisions and the company will be harder to manage physically \autocite[39]{Galbraith197129} due to changing team sizes and locating co-workers.

\subsection*{When do you need a Matrix Organization?}
As described in the previous section, the matrix organization introduces overhead and complexity. There has to be substantial motive to transform the existing organization form to a matrix.
In a functional organized company you may want to move towards a matrix organization if the company requires faster response to market demands. If these market demands are not as urgent, the company may want to refrain from a matrix organization because of this introduced complexity and overhead.

The same holds for choosing to move from a project organization to a matrix organization. The benefit of the matrix organization is that it will bring more specialization and better utilization of personnel. However, if the market of the organization does not require better specialization and/or resource allocation is not too difficult (not dealing with short staff on projects), then the company may want to stay a project organization.
The choice is not easy because the effect is unforeseeable and I did not encounter any numbers showing the effect of introducing a matrix organization form. As the literature states, the choice is mainly a matter of required change \autocite[70]{MEE196470}. I think the hard part will probably be the timing.

\subsection*{How to start using a Matrix Organization?}
A matrix organization will significantly change the structure of the organization. The main changes will occur at the staff level. The sea level of the organization will be less exposed to the changes. \autocite[113]{KNIGHT}
Due to the change on staff level, the change will impact the biggest group of employees. Staff will have more communication and will get more responsibility (moving from a functional organization). Therefore, it is key that:

\begin{enumerate}
\item The staff members know why the organization is changing towards this organizational form.
\item The staff members know how it will affect their workflow.
\item The staff members are (as much as possible) content with the change.
\end{enumerate}

To allow the staff to adapt more easily to the change, an organization may decide to move towards a matrix organization in phases \autocite[39]{Galbraith197129}.. This can be composed to `moving stepwise to the spectrum' as mentioned in \textit{What is a Matrix Organization?}.
Another challenge that arises from a matrix organization is the fact that staff members will be part of more groups than in a pure functional or project organization. Also, the type of people will vary. For example, in a functional organization, people from the same division are likely to have the same interests and point of views. When moving to a matrix organization, new groups are formed (by force), increasing the chance of conflicts. Therefore, it may be important to increase the focus on people skills of both the staff and the managers of those groups.

Not all organizations will fully thrive in the matrix organization. For example in a project organization, there might be an `innovation team'. This team will work on new concepts and may have a uniquely qualified person of the organization as member. This member may not fit in a division (vertical slice) in the matrix organizations, while his team members do.

\subsection*{Unanswered Questions}
The literature left some of my questions unanswered.

First of all how performance of teams will be measured. Within a functional organization it was clear if a division was performing well by analysing both input (budget) and output (reached targets). In a matrix organization it is much harder to measure this, because members of one division are located in different teams that probably perform differently. Not knowing how your divisions perform may be a risk to the quality of specialization of the employees.

Another question aspect is the notion that the matrix organization brings up the `best of both worlds': maintain the specialism of the functional organization and the fluidness of a project organization. I foresee the risk that you may end up with inadequately qualified personnel that are still not able to meet the deadlines.

\subsection*{Conclusion}
The matrix organization form can add value to an organization when the existing form is not sufficient. It also adds overhead and complexity to an organization, which may be compensated by the market advantage. An organization will change significantly in form of communication and responsibility, a change which needs to be handled appropriately.

Personally I think the matrix organization is only effective when it is a necessity: for example, an organization grown into a size where the current form is inadequate. I believe the matrix organization is a good form since it lets an organization maintain their original mission. A functional organization would be able to reach the excellence they previously achieved with specialized personnel.

In IT this organizational form seems an appropriate form for companies such as Spotify. Companies that require specialization (due to the size and uniqueness), and have to change continuously to fight the competition.