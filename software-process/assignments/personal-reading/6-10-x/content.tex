\section*{What is 10x?}

10x is one of the oldest and most researched topics in software engineering where measurments are applied ot identifie if a 10-fold productivity holds between different programmers \autocite[567]{MAKING_SOFTWARE}.
These measurements are in place to verify if programmers have an higher productivity when having the same level of experience.
The `10x' is more a figure of speech than an actual multiplication of performance.
It is mainly used to show a significant difference in magnitude between two object.
As expressed by Bossavit, ``the 10x claims has been around forever'' \autocite[37]{bossavit2013leprechauns}.
In this expression, as Bossavit mentions himself, you should read `forever' as since the beginning of the software engineering as an academic discipline \autocite[38-39]{bossavit2013leprechauns}.

In my opinion this is not strange, while in business always strives to increase productivity, managers want to know where the money goes.
DeMarco and Lister state the measurement of individual productivity as a point of friction [268, DeMarco].
They state that ``the presumption that there are order-of-magnitude differences in individual performance makses cost porjection seem nearly impossible'' [268, DeMarco].
As a variation on the `10x' research, DeMarco and Lister focused on the permance of workplaces rather than the performance of individuals.
More on this in the section \textit{Measurement}.

\section*{What do we know?}

In 1968 Sackman and Erikson published an article on the difference in performance performing different programming tasks \autocite{sackman1968exploratory}.
The main findings in this article are \autocite{sackman1968exploratory} \autocite[567]{MAKING_SOFTWARE}:
\begin{itemize}[noitemsep]
\setlength{\itemindent}{-.2in}
\item The ratio in coding time between the best and worst programmer was 1 to 20;
\item The ratio in debugging time between the best and the worst programmer was 1 to 25;
\item And the ratio of the size of the program between the best and the worst programmer was 1 to 5.
\end{itemize}

This paper triggered a lot in the software engineering community.
And as Bossavit shows, there were multiple reproductions on the original experiment \autocite[38,41]{bossavit2013leprechauns}.
The main part of these studies could not reproduce the original results from Sackman and Erikson \autocite[42-44]{bossavit2013leprechauns}.
The other part of the studies had inconclusive results.

As McConnel summarizes nicely in his chapter on `10x' in the \textit{Making Software} book \autocite{MAKING_SOFTWARE}:
``The general finding that `There are order-of-magnitude differences among programmers' has been confirmed by many other studies of proffesional programmers'' \autocite[568]{MAKING_SOFTWARE}.
The studies that McConnel cites do show that thare is this difference in magnitude, but they differ over the different studies.

So there are differences, but the question remains what can these differences mean.
Augustine observed that more than 50\% of the work is typically done by 20\% of the people \autocite{augustine}\autocite[268]{demarco1985programmer}.
This may be arguably correct, but as presented in the next section: it depends on the measurement.

\section*{Measurement}


TODO - variations in measurement
TODO - Why is it effective
TODO - maybe oppertunistic? character, arguable that 10x, but not effective in industry
TODO - demarco with work envorinment. Show great influence on individual state/form

\section*{Is it effective?}

TODO - Environment is never the same.
TODO - Making software, you need a goal
TODO - Augustin, but what about the quarterback?
TODO - Counter effective?
   > Making software slacker
   > Good results
   > A good programmer, which is actually not measured correctly does not get his appreciation.
   > Moves to a company where he is 'valued' a.k.a. oppionion respected, listening and resulting in bettter communication and better code?
TODO - Database developer - is java

TODO - Wife, 2 kids or 28 and living in his parents basement

\section*{Is it real?}

TODO - I believe it is real, just a gut feeling.
TODO - Measurement is just too hard due to the environment, as we can see from demarco's work.
TODO - Making software tries to make an argument on lotus/excel. Hire policy and env?

\section*{Conclusion}

TODO - Questionable if we can say something.
TODO - The organization and the project form make the difference.

