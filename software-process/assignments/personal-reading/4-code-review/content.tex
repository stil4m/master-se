\subsection*{What is Code Review?}

Code Review is a part of a development process where developers examinate code to find defects \autocite[47]{10.1109/MS.2003.1241366}.
This process is conducted by the company developing the software. Not an external party.
Companies apply Code Reviews to deduce the amount of defects in their code base in an earlier stage [TODO].
Code reviews, also called peer reviews, catch an average of 60 percent of the defects of the reviewed code base \autocite[136]{10.1109/2.962984}.
Different studies have shown that defects in a later stage of a project tend to cost more money to resolve than in earlier phases \autocite[135]{10.1109/2.962984} \autocite[21]{beck2000extreme}

\subsection*{What kind of Code Review Exist?}

Different types of code review exist. For example \autocite[23--38]{cohen2006best}:
\begin{itemize}
\setlength\itemsep{0em}
\item \textbf{Formal Inspections} \autocite[23]{cohen2006best}\\
A formal inpection code review is a meeting where developers come together and review a piece of software.
For example, the code will be presented on a beamer and the developers will note corrections when they find defects.
The found defects can be formalized and applied after possible discussion.
\item \textbf{Over-the-Shoulder Reviews} \autocite[26]{cohen2006best}\\
An over-the-schoulder review is a review session where a colleague inspects a piece of changed software with the guidance of the programmer that did built it.
The developer will talk his colleague through the made changes and feedback can be easily addressed or directly fixed in the code base.
\item \textbf{E-mail Pass-Around Reviews} \autocite[30]{cohen2006best}\\
A (lead-)developer will gather changes on a piece of software for a certain time-frame and wil send them around in email to a group of developers.
Developers can reply on the email with the defects they have found.
\item \textbf{Tool-Assisted Reviews} \autocite[34]{cohen2006best}\\
A tool is employed to assist with the code review.
When developers want to perform a code review, they use the tool to invite other developers to perform a review.
The tool will support the changed files and the changed content in a convenient manner and the other developers can comment on the changed code within the tool.
\item \textbf{Pair-Programming} \autocite[37]{cohen2006best}\\
Pair-Programming is a practice where two developers work on a single task behind a single machine.
One programmer will do the actual programmer (the driver), while the other developer (the navigator) will continuously check the code and gives feedback.
This feedback is a form of code review, but also contains additional benifits such as learning from each other.
\end{itemize}

Cohen et al. describe these forms, but I can imagine there are additional forms.
For example, on friday afternoon with the delight of a beer you can join with a view collegues to review a certain module. This may be an informal alternative of Formal Inspection.
Some of these five types are cost-intensive, such as Formal Inspections and E-mail Pass-Around Reviews.
The main cost will be addressed in the preparation, multiple people dedicating their time and collecting the necessary information \autocite[23--38]{cohen2006best}.
One statement of Cohen et al. I found a bit odd.
They state that E-mail Pass Around Reviews ``is the second-most common form of informal code review, and the technique preferred by most open-source projects'' \autocite[30]{cohen2006best}.
I have worked in open-source projects and the main code collaboration tooling such as
BitBucket\footnote{https://bitbucket.org/},
GitHub\footnote{https://github.com/},
GitLab\footnote{https://about.gitlab.com/}
and FishEye\footnote{https://www.atlassian.com/software/fisheye}
all have Tool-Assisted Reviews.
Cohen et al. published their work in 2006, I believe that last ten years changed a lot in this area.

I have never attended a Formal Inspection but can imagige, just as Cohen et al. address, they are of high costs due to the high number man-hours allocated for the task.
Votta et al. showed that around 96 percent of the defects could be found when the same individuals reviewed the code on their own \autocite[110]{Votta:1993:INM:256428.167070}.
Due to the unknown, how much money the remaining 4\% may cost it may be cost-saving when using Tool-Assisted Reviews instead of Formal Inspections.

In my experience in the industry I have only worked with Pair-Programming and Tool-Assisted Reviews and a little with Over-the-Shoulder review. Personally I think that the others are a bit inconvenient.
For example, for Formal Inspection you have to gather a group of developers on a certain time.
I know my collegues and none of them every `has time' and
additionally with Formal Inspection you will force a developer to stop his current tasks and switch context on possibly an inconvenient moment.
This together with the findings from Votta presented in the previous paragraph may show that Tool-Assisted Reviews are more convenient than Formal Inspections.
Tool-Assisted Reviews allow developers to allocate time when it fits their workflow, which I personally find important while switching context (from your code to someone elses) is hard.

The same holds for E-mail Pass-Around Review. There is a workload to gather all changes, send them to colleagues wait for responses and than gather them all together.
This seems to be an administrative exspensive task and error prone.

I wanted to make a last remark on these types of code review related to Over-the-Shoulder Reviews.
I have only encountered these for bug fixes that had to be deployed on a production environment as fast as possible.
These fixes are often just a one-line patch.
Because a formal review was too time consuming and the context was too critical to just commit the code an extra pair of eyes would validate the patch.
In my opinion this is an appropiate form of code reviewing whereas the situation.

\subsection*{Merits and Drawbacks}

As described earlier, code reviews have merits and drawbacks.
The main advantage is the elimination of defects in an early phase, which gains a decrease in cost as mentioned in \textit{What is Code Review?}.
This is directly contradicted with the fact that people have to spend time on the code review.
It is unknown if these outweigh each other or if is an umbrella for the other.

The researched literature omits some of the advantages that I actually find pretty obvious.
The encountered information mainly focuses on the price tag and the design of code reviews.
Additional befefits that I see are:

\begin{itemize}
\setlength\itemsep{0em}
\item
For low fault-tolerant systems it is additional manner to find defects.
\item
When reviewing somebody else his code, there is automatically co-ownership of the code.
At least two developers in the organization know about the change.
This can benifit the organization when the first developer gets sick or even leaves the company.
\item
Code reviewing will help with code-style. When developers write their code in different styles, a code review is an optimal process to discuss these divergents.
One can argue that these style issues are covered in a style-guide, but naming conventions such as \textit{findEntity} or \textit{getEntity} for function names are sometimes not covered.
Code reviews can also help to expand the style-guide.
\item
It is a learn and control mechanism for junior programmers, interns or new team members.
Inexperienced programmers can learn from more experienced programmers by reading their code.
Also, the experienced programmers can check if no invalid code is added by the inexperienced developers.
This may be benefical in an complex domain such as finacial regulations.
\end{itemize}

I think these additional benefits may prevail the choice to start using code reviews.

\subsection*{When use Code Reviews?}

TODO - Not clear. Not clear from cost persp.
TODO - If one of the points hold.
TODO - Try
TODO - Personally think smaller teams.

\subsection*{How to apply Code Reviews?}

TODO - 400 lines mark
TODO - 60 min mark
TODO - Personal experience

\subseciton*{Conclusion}

TODO - Practice to decrease defects.
TODO - Cost unknown
TODO - Brings Non-obvious advantages
TODO - Start directly in quality environment