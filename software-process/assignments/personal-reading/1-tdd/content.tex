\subsubsection*{}
For this weeks reading assignment I have chosen Test Driven Development (TDD) as topic due to two reasons. First because
it was one of the suggested topics and secondly because I have some experienced from when I applied it in a 6 month project
and wanted to reference these experienced to scientific research. The main literature for this topic I used is the 'Making Software' \autocite{MAKING_SOFTWARE}
book chapter 12 and the paper 'A structured experiment of test-driven development' \autocite{SAC-J-2003-GeorgeW04}.
The book mainly elaborated on the definition of TDD and presents the results of empirical studies. The paper presents a
small scale empirical study and a survey on the experience on practicing TDD.

\subsection*{What is Test Driven Development}

As the Making Software book presents, TDD is a software development practice where a repetition of steps should lead
robust code. These steps are \autocite[208]{MAKING_SOFTWARE}:

\begin{enumerate}
  \item Choose a small task.
  \item Write a test for that task.
  \item Run all the tests to verify that the new test fails.
  \item Write minimal production code to complete the task.
  \item Run all tests (including the new one) to verify that they pass.
  \item Refactor the code as necessary.
\end{enumerate}

In general the literature is written with the notion of unit testing as a tool to perform TDD. This complies to the
first step 'Choose a small task', which may correspond to a testable unit. In the book the writer writes about
deviations on TDD mentioning Behaviour Driven Development (BDD) and Acceptance Test Driven Development (ATDD). These
are both extensions on TDD that respectively provide a better framework to work with and connect the business
expectations closely to the tests. Neither of the referenced literature mentions the usage of different test types,
such as using integration tests and/or system tests. I will reflect my point of view later on in the section
'Approach'.

Based on the literature the effectiveness of TDD is uncertain. The results that arise from the performed empirical
studies are not conclusive that it is likely that practicing TDD will improve a a project in aspects such as:
productivity, internal quality, external quality and test quality \autocite{SAC-J-2003-GeorgeW04} \autocite{MAKING_SOFTWARE}.

\subsection*{Merits and drawbacks}

As all the things in life, also TDD has it's merits and drawbacks. The main advantages that the literature presents
as arguments in favor for this method are:
better external quality \autocite[340]{SAC-J-2003-GeorgeW04} \autocite[212]{MAKING_SOFTWARE};
TDD utilizes the understanding of code and thus helps to manage complexity \autocite[338]{SAC-J-2003-GeorgeW04};
creates test assets that utilize (automated) regression testing \autocite[338]{SAC-J-2003-GeorgeW04} \autocite[208]{MAKING_SOFTWARE};
and decreases the chance of defect injection \autocite[338]{SAC-J-2003-GeorgeW04}.

\paragraph{}
Some of these arguments are are presented clearly with scientific foundation, such as the increase of external quality
when TDD is applied. This effect is pretty straightforward when you study the definition that 1) all new code is
tested while you are not allowed to write more code than needed (point 4 in the definition) 2) the new code does not
break the existing implementation (point 5 in the definition) and 3) the implementation is always feature centric.
This implies that by definition it is impossible to introduce code that degrades the external quality, because all
code is introduced intentionally by the developer as he realizes requirements. This statement is verified by the
advantage that the defect injection chance is decreased.\\
A what I find more obscure advantage is the statement that TDD utilizes the understanding and management of
complexity. Firstly, this point is not verified with scientific proof in the researched paper. Secondly as the paper \autocite[338]{SAC-J-2003-GeorgeW04} addressed and the book lacks to do, using TDD does not lead to formal documentation (while the test is your
documentation). The alternative to documentation in TDD are the produced tests, however the required information may
not be presented to the reader in a glance. For example: \\

\begin{itshape}
\begin{addmargin}[1em]{1em}
Suppose there is a functionality `A` with the specification 'When a number is entered between 1 and 10, multiply it by
2 otherwise throw an exception. Using TDD this will result in at least 3 tests: testing a valid number; testing a
number beneath 1; and testing a number above 10. When a developer has to search for the specification based on the
tests he has to look into the details of 3 tests, which possibly will not fit into one developer console due to
programming language syntactics. Instead a simple comment above the implementing code or a formal requirement may be
more helpful to the developer.
\end{addmargin}
\end{itshape}

\paragraph{}
The above example is rather small and in my opinion will cause more problems when functionality grows when there is a
lack of documentation. The reference implementation I got for this problem is an algorithm for medical images to
transform a 16-bit gray-scale image to a 8-bit gray-scale image with additional parameters to shift the domain and the
window on the image. The tests written using TDD were created when I knew the domain and the requirements of the
algorithm. However when you open the test code at this moment it is not possible to figure out what the specification
was. The point here is that you can not assume that the test code is a replacement for formal documentation.

\subsection*{Reliability}

When applying TDD to build a program or software module, the building blocks are tests. The literature states that if
you apply TDD the amount of building blocks grows, which leads to certainty as you continue building in the form of
regression. I believe that this is true in a certain point of view, the books point of view. But I would like to shed
light on the reliability of these building blocks with a simple example. \\

\begin{itshape}
\begin{addmargin}[1em]{1em}
With unit testing a tester focuses on testing the external quality of a single unit. To correctly test a single unit
the tester should 1) remove all dependencies on IO such as file system interaction and network activity 2) take
control over the instances the unit depends on, performed by mocking dependencies \autocite{XP_EXAMINED}, 3) improve
testability by introducing patterns such as dependency injection. Suppose the
tester fully tests functionality `A' and mocks a functionality `B' on which `A' relies When the tester
mocks the behaviour of object `B' he should know what the domain and range of the behaviour is to create correct
tests for `B'. So far, so good. The problems arise when there is a request for change on functionality `B'
whereas the domain and/or range changes, a developer changes `B' and updates the functionality and its related
tests to the new specification. There will still be 100% test coverage if both implementations are truly developed
with TDD and all tests will pass, but with this example we can conclude that functionality `A' is not tested
properly i.e. it does not integrate properly with functionality `B'.
\end{addmargin}
\end{itshape}

\paragraph{}
The above example shows that pure TDD development by using the notion of unit tests may lead to the assumption that a
system is tested properly since most components are tested, however bugs may show up in both tested as untested code.
The point is that the foundation on which TDD is build, might give a wrong sense of confidence. Although the outlined
problem can be simply solved by introducing integration tests or system tests, this contradicts with the definition of
TDD 'Choose a small task'. The developer will write considerably more code before all tests pass.

\paragraph{}
The mentioned confidence is completely inappropriate. The following statement will hold 'When the tests do not
pass, then the code is not correct.', however this does not imply that the inverse will hold: 'When the tests
pass then the code is correct'. Developers applying TDD must be aware that the inverse of the original statement is a
dangerous assumption.

\subsection*{Approach}


A developer can build functionality with or without TDD in two directions: bottom-up, or top-down. For example when
requesting a user from a webservice, the developer may start at the endpoint in the server and work its way down, or
start at the database and work its way up. From my personal experience I encountered that you keep rewriting your
upper layered implementations because you have encountered that you are missing information in one of the lower
layers or vice versa. By using TDD the developer needs to rewrite both implementation and test. It is arguable if I
have enough experience, but as both the paper and the book state TDD will lack upfront design \autocite[338]{SAC-J-2003-GeorgeW04} \autocite[212]{MAKING_SOFTWARE},
resulting in a
greater chance that a developer can't foresee all the aspects of the detailed implementation. I believe that this both
happens in top-down as bottom-up workflows.

\subsection*{Questionable Aspects}

The research show different aspects of the effects of TDD. Some interesting aspects I want to outline are summarized below.

\subsubsection*{Productivity}

The research clearly points out that the productivity decreases (a developer requires more time to develop new
functionality) when applying TDD, this is understandable when you realizing he developers may actively spend more time
on testing than when developing in a code-first style. This is also shown with the number of tests and the code
coverage related to the test quality when applying TDD \autocite[341]{SAC-J-2003-GeorgeW04}. A question which remains unanswered is does this
productivity remain lower in the overall picture? A project where TDD is applied may have less bugs than a system
developed without TDD. The absence of creating bug fixes might make up for this decrease in productivity. I did not
find related literature which proves or disproves this.

\subsubsection*{Level of difficulty}

Both the paper and the book qualify TDD as hard or difficult. Personally I believe, this is not true and the
difficulty statement is context dependent. I believe mocking input/output, web-requests and database behaviour is hard
i.e. the side-effects and when a developer starts applying TDD he faces these concepts. This does not make TDD hard,
but only the concept of unit testing. But that was also hard when the developer applied the code-first style. However
with this style the developer was free to skip testing functionality in full extend. This can be addressed as lack of
discipline, which brings me to my point to make: Discipline is hard, not TDD. TDD requires discipline, while
code-first allows developers to cheat. This is may be why developers find TDD hard, they lacked the discipline in the
first place to write properly tested code in their code-first style.

\subsection*{Conclusion}

As the literature shows Test Driven Development improves external quality of code. The increase of test quality
contributes to this aspect and allows a developer to be more confident, but when carried out too much may result in a
unfair amount. As I have presented with the different examples, TDD is not replacing documentation, is build on top of
an incomplete practice (unit testing) and requires a lot of discipline. Therefor I conclude that the concept of TDD is
strong, but still too weak. For example I think it can be vastly improved by writing integration tests up front.
