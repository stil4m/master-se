\paragraph{}
\textbf{
Abstract - This is an essay on the paper 'A Metrics Suite for Object Oriented Design' by S. Chidamber and C.
Kermerer which is dated from June 1994. Firstly there will be a brief summary about the paper itself followed by
its merits and drawbacks.
}

\paragraph{}
When object oriented design (OOD) starts to unfold in the industry, the need for feedback on this kind of software
development increases. The writer addresses the problems that cause the need for a metrics suite that provides the
required feedback on OOD. The existing metrics lack the theoretical ground, are not sufficient of OOD,
technology dependent or too labor-intensive to collect.
The paper introduces six metrics as a suite: 'Weighted Methods Per Class' (WMC), 'Depth of Inheritance Tree' (DIT),
'Number of Children' (NOC), 'Coupling between object classes' (CBO), 'Response For a Class' (RFC) and 'Lack of
Cohesion of Methods' (LCOM). These metrics are assessed with the software complexity metric properties introduced by
Weyuker [TODO cite weyuker] and found to comply with most of the specified properties.
Criticism on previous research was the lack of theoretical basis which is refuted on this paper by the use of the
model provided by Weyuker and lack of empirical data, which is added by applying the metrics on two different OO
applications (C++ and Smalltalk) using an automated tool. The main contribution of the paper can be summarized as a
suite of six individual software complexity methods that can be combined to illuminate design problems and/or choices
within an OOD.

\paragraph{}
The paper is well structured and meets the expectations from the reader. Where the writer introduces the problem
(section II), he succeeds this with separate sections on the point of view and approach. First the writer show how he
approaches the properties introduced by an OOD and which metrics were abstracted (section III), secondly he shows how the
lack of theoretical ground is tackled in this research (section IV) and ends this with a presentation of the empirical
data collection (section V). Section VI contains the research results. The writer iterates over different metrics and
accompanies each metric with a point of view, performs an analytical evaluation and presents the empirical data and its
interpretation. This section concludes with a summary. The paper ends with the concluding remarks.

\paragraph{}
While the paper does not have a research question it has a clear research problem, which is actually tackled.
The problem is divided in two parts. The first is the lack of metrics that relate to the properties that emerge from an
OOD, which indicate the need for such metrics. The second part concentrates on the theoretical criticism of metrics
applied to conventional non-OO applications. The second part is an argument to include a theoretical base in the
research, but the provided substantiating research introduced by the writer merely focuses on complexity metrics instead
of metrics in general. The writer lacks of explicitly scoping the research to complexity metrics, but abides to this
point of view throughout the rest of the paper. With the constructive use of complexity as point of view and the use of
a theoretical basis that is based on the complexity measures by Weyuker, the writer should have addressed this scope in
the title, abstract or introduction of the paper. Also the writer does not introduce the the concept of complexity. The concept of
software complexity could be interpreted in different ways with metrics introduces by McCabe [TODO cite McCabe]
and Henry [TODO cite Henry].

\paragraph{}
The writer argues, as mentioned above, that existing metrics do not relate to the properties of OOD, but the writer fails
to demonstrate this or refer to previous works that indicate this. Although the reader can assume the lack of these
metrics due to maturity of object oriented development in the industry at that time, the writer should have introduced
his assumptions or add references to previous work that show this. The previous works that are referenced include suggestions
and attempts for OOD metrics, but these sources do not address the missing relation of existiting metrics with the OOD
properties, nor does the author.

\paragraph{}
In the most important section of the paper the writer iterates over each proposed metric. The writer applies the
same pattern for each metric, which implicitly sets the expectation for the reader. For each metric the writer gives a
clear definition, theoretical basis and his point of view. The description and the theoretical basis give the definition and
the motivation for the metric. The point of view of each metric adds a more vivid argument for the need of each metric.
Some of these arguments are more application-type specific. For example: ``The larger the number of methods that can be
invoked from a class, the greater the complexity of the class.'' I cannot reason about the types of application that were
developed at the time of the paper, but in the present day there are a lot of APIs that separate their data from their
logic. Although this is a more procedural approach than OO, it is applied with OO languages. The method count
with these designs is higher in the data objects while they tend to be much simpler than the code in the domain-specific
layer of an application. Aside from a few questionable point of views, the arguments the author makes are solid.
Where the point of view of the paper is convincing, the theoretical basis lacks detail on a few points. For example:

\begin{itemize}
\item
The definition of the NOC metric states: ``number of immediate subclasses subordinated of a class in the class
hierarchy'', but the theoretical basis states: ``it is a measure of how many subclasses are going to inherit
the methods of the parent class''. The number of inheriting classes is not equal to the immediate subclasses of a
class.
\item
The LCOM metric is a computation based on the intersection of accessed fields of two methods. To identify two
methods with or without accessed field similarity, the writer uses the notion of pairs. These pairs however are
computed using the Cartesian product of a list containing field access sets (which are mapped from the
methods). The Cartesian product holds the reflexive relational property, which by definition will decrease
the outcome with at least the number of elements in the set consistent of the field access sets.
The theoretical basis of LCOM states ``the larger the number of similar methods, the more cohesive the class'',
but a simple counter example shows that there is a bug in the definition. Assume we have a set of instance
variables: $\{I_1, I_2\}$ and a set of methods $\{M_1, M_2\}$ where $M_1$ uses $I_1$ and $M_2$ uses $I_2$, then from
the LCOM definition, $P = \{( \{I_1\}, \{I_2\} ), ( \{I_2\}, \{I_1\} ) \}$ and
$Q = \{( \{I_1\}, \{I_1\} ), ( \{I_2\}, \{I_2\} ) \}$. This results in a LCOM value of $0$ that indicates that
the class is most cohesive, but $M_1$ and $M_2$ do not relate in a single matter. This problem is presented in other
works such as Basisi et al. [TODO cite basili] and Chai et al. [TODO cite chai]
\end{itemize}

\paragraph{}
These points show that some metrics are not sound in a manner that there is no symbiosis between the definition and
the application.

\paragraph{}
After each metric is made clear the writer gives an analytical evaluation of the metric and presents the empirical
data. The analytical evaluation is a formal proof, which indicates that each metric complies to the metric properties
specified by Weyuker.

\paragraph{}
The writer summarizes the result in the last part of the paper, but also introduces new concepts on how the metrics
combined could be used as design indicators, why some properties of the metrics model are not met and why, and the
future direction of research. I rather have seen this in separate sections, mainly due to the reason that these
components deserve their own section. The introduced point of views of combined metrics is one of the main contributions
of the paper as it has a closer relation to OOD than the individual metrics.

\paragraph{}
To conclude, the paper is well structured and has a clear contribution. The research problem is clear, but the
approach tends to focus mainly on complexity metrics and has some aspects that could be described more explicitly, and it
has some inconsistencies in the metric definition and application. However the presented data is solid.
As intended, the metrics can be used as indicators to prevent faults as Basili et al. [TODO cite basili] showed.
